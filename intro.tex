\section{Intro}

Creating games often proves challenging from both a design perspective and a programming perspective []. 
In order to simplify the problems at the programming side of this challenge, a new programming language was created. 
The programming language (Casanova) is designed to better express the abstractions found in games. 
During development of Casanova the research team found that the code of the Casanova compiler became difficult to maintain. 
To address this issue, a new language was created: Meta Casanova.

\subsection{Context}

\subsubsection {The company}

The graduation assignment is be carried out at Kenniscentrum Creating 010. 
The company is located in Rotterdam. Kenniscentrum Creating 010 is a transdisciplinary design-inclusive Research Center enabling citizens, 
students and creative industry making the future of Rotterdam[].

The assignment is carried out within a research group, who is building a new programming language. 
The new programming language is called Casanova.

\subsubsection {The research group}

The research group that focuses on the Casanova Language comprises of the following members: 
Francesco di Giacomo, Mohamed Abbadi, Agostino Cortesi, Giuseppe Maggiore and Pieter Spronck.

In the research group there is a team that works on developing Meta Casanova and its compiler. This team consists of interns, namely:

Jarno Holstein, that works on the front-end and back-end of the compiler.

Louis van der Burg, who develops and debugs the MC language.

And Douwe van Gijn, who develops the back-end of the compiler.

\subsubsection {Motive}

Meta Casanova is a programming language meant for the development of compilers. 
The need for such a language arose after the research group for the Casanova programming language found that their compiler became difficult to maintain and adapt. 
This difficulty originated from the fact that F\#, 
the language Casanova was written in, did not support the higher level of  abstractions that compiler developers could benefit from. 
Thus MC was created. 
MC proved itself preferable over F\# when the Casanova compiler was rewritten in MC. 
What used to be 1480 lines of code in F\# are now 300 lines in MC []. 

The MC compiler does however have a few drawbacks. 
The compiler does not give useful error messages, if something is wrong in the MC code it will be hard to identify exactly what. 
It is difficult to maintain and expand the compilers code. 

So when a new and improved version of MC was developed, namely MC mark 3,  
the research team decided to also fix the drawbacks that the compiler was facing by developing a new compiler. 

