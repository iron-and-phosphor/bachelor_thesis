\input{markup.tex}

\begin{document}

\title{Designing and implementation of the Meta Casanova 3 compiler front-end}
\author{\writer}

\begin{titlepage}
   \input{voorpagina}
\end{titlepage}

\begin{abstract}
   \emph{Designing and implementation of the Meta Casanova 3 compiler front-end.}
   The internship will consist of developing a maintainable front-end and type checker for the bootstrap compiler of Meta casanova mark 3.
   The goal of the assignment is to develop a system for the front-end of the compiler that can later still be easily maintained by other developers.
   This internship will be executed within the language design field of computer science.
   The internship will be carried out at \emph{Kenniscentrum Creating 010}.
   The research will be qualitative and will make use of descriptive and testing research methods.
   The end result will consist of a design and implementation for the front-end and type checker. 
   The documentation wil be in the form of a bachelor thesis.
\end{abstract}

\section{Introduction}
\subsection{Working title}
Designing and implementation of the Meta Casanova 3 compiler front-end.

\subsection{Motive}

Creating games often proves challenging from both a design perspective and a programming perspective []. 
In order to simplify the problems at the programming side of this challenge, a new programming language was created. 
The programming language (Casanova) is designed to better express the abstractions found in games. 
During development of Casanova the research team found that the code of the Casanova compiler became difficult to maintain. 
To address this issue, a new language was created: Meta Casanova.

Meta Casanova is a programming language meant for the development of compilers. 
The need for such a language arose after the research group for the Casanova programming language found that their compiler became difficult to maintain and adapt. 
This difficulty originated from the fact that F\#, 
the language Casanova was written in, did not support the higher level of  abstractions that compiler developers could benefit from. 
Thus MC was created. 
MC proved itself preferable over F\# when the Casanova compiler was rewritten in MC. 
What used to be 1480 lines of code in F\# are now 300 lines in MC []. 
\linebreak
The MC compiler does however have a few drawbacks. 
The compiler does not give useful error messages, if something is wrong in the MC code it will be hard to identify exactly what. 
It is difficult to maintain and expand the compilers code. 
\linebreak
So when a new and improved version of MC was developed, namely MC mark 3,  
the research team decided to also fix the drawbacks that the compiler was facing by developing a new compiler. 


\subsection{Importance}
The result of this thesis will contribute to the development of Casanova and MC.
The thesis will also serve as documentation for future developers of MC.

\subsection{Goal}\label{sec:goalsmandate}
The goal for this project is to create a maintainable and extendable front-end and type checker that future developers can continue with.

\subsection{Problem statement}


\subsection{Central research question and sub-questions}
The following research question could be formed of the goal. 
   How to develop a maintainable and extendable front-end and type-checker for the MC mark 3 programming language?
In order to answer the main question the following sub question were formed.
   What properties does MC have that the front-end needs to process?
   How to develop a maintainable and extendable parser for MC?
   How to apply a type-system to MC?
The project comes with the following requirements:

Adaptability requirement
The code created during this project should be easily maintainable by developers who did not initially design the structure of the compiler.

Correctness requirement
The program is a compiler and has an even smaller margin for bugs than other programs have. 
No program the compiler compiles would be bug-free if the compiler itself is not bug-free.

Diagnostic requirement
The front-end and type checker need to be able to give descriptive error information for the programmer of MC.

Central research question and sub-questions
From this problem statement follows the following research question:

Diagnostic requirement
The front-end and type checker need to be able to give descriptive error information for the programmer of MC.

\subsection{The client}\label{sec:clientmandate}
The graduation assignment will be carried out at Kenniscentrum Creating 010 with the research group of MC.
The company is located in Rotterdam.
\textit{Kenniscentrum Creating 010 is a transdisciplinary design-inclusive Research Center enabling citizens, students and creative industry making the future of Rotterdam}~\cite{creating2016home}.

The research group is active within the language design area of computer sciences.


\subsection{Working environment and tasks}\label{sec:workenvmandate}
During the internship the student will work as a part of the existing research team, whom do research
in the Casanova and MC languages. The student will be doing research and programming
and will be supported and helped by the entire research team. Every two weeks the student will
present his progress and receive feedback.

The student also needs to show several competences:

The competence \emph{administering} will be attained by developing the front-end and type checker according to the requirements set by the research group.

The competence \emph{Analysing} will be attained by analysing the Meta Casanova language and creating a proper front-end for the language.

The competence \emph{Advising} will be attained when communicating with the stakeholders.

The competence \emph{Designing} will be attained developing the front-end and type checker.

The student will be supported and helped by the entire research team.


\section{Method}\label{ch:methodmandate}
\subsection{Research methods}
During the internship there will be made use of different research methods. This stems from
the wide scope of the project and the fact that there are multiple research questions which
require different approaches.
First there will be research into lexers, parsers and type checkers. What are the possibilities
and what have others already accomplished.
Then exploitive research will be held to find a proper way to implement the MC language in
the compiler.
And finally a explanatory research will explain the working of the front-end and mid-end of
the compiler.


\subsection{Gathering information}
The information needed will be gathered from the research team involved, via personal interviews,
and via published papers concerning the material.


\subsection{Validation of results}
The validity of the results will be secured by regularly discussing with the client and the supervisor.
Furthermore the results will be viewed as correct if the front-end is able to parse the
features of the MC language. The type checker will be correct if it can produce a data structure that
the back-end can use to generate code.

\subsection{Validity and trustworthiness of sources}
The sources used will be published papers and books respected in the compiler community.
Other sources, such as personal communications will be with people who are approved by the
supervisor.

\subsection{Project methods}
During the internship the LEAN-software development method will be used\cite{ries2011lean}.
Using LEAN gives the ability to quickly iterate through versions and gather knowledge more easily through these iterations.

When using LEAN results are fast available.
These results can be used to catch faults early on and adjust the project accordingly.


\subsection{Risk analyses}
\begin{center}
   \begin{tabular}
      {| p{0.2\textwidth} | p{0.25\textwidth} | l | p{0.3\textwidth} |}
      \hline
      \textbf{Risk} & \textbf{Effect} & \textbf{Possibility} & \textbf{Counter measure}
      \\ \hline
      The language changes during development of the compiler. & Parts of the compiler need to be rewritten or addapted. & 40\% & Make the compiler flexible enough to accommodate future changes.
      \\ \hline
      The interface with the back-end changes. & Data that may not be available would be needed. & 60\% & Keep as much data available in the front-end and type checker.
      \\ \hline
      There is not enough time to add all the MC features to the compiler. & The compiler will not be able to compiler the full language. & 15\% & Asses which features are more important then the others and implement those.
      \\ \hline
   \end{tabular}
\end{center}

\subsection{Quality expectations}
The lexer and parser are able to parse the MC language and give descriptive error information to
the user. The type checker needs to generate correct type information that can be used by the
Back-end.

\section{Results}
\subsection{Intended result}
The end result should be a front-end and type checker that is easy to maintain and expand upon.

\bibliographystyle{ieeetr}
\bibliography{biblio}

\begin{appendices}
   \input{stakeholders}
\end{appendices}
\end{document}
