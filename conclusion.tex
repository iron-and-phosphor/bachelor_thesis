\section{Conclusion}

In this thesis, we have seen what information was gathered for the development of the front-end and type checker. We looked at the syntax of MC, then we took a look at how parsers in general work which led to information on the parser monad. We saw how the parser for the MC mark 3 compiler works and found that the requirement for this project where met.

At the beginning of this project there was a research question: How to develop a maintainable and extendable front-end and type-checker for the MC mark 3 programming language?

During this project we found that the answer to that question is that the MC syntax allows for the front-end to be separated into several modules that allow for either parsing or type checking. Because the front-end is split into multiple modules the code of the front-end becomes more organized and easier to maintain. If functionality needs to be added to the front-end or type checker another module can be created and added to the system.

The code developed for the front-end, along with the documentation will further the goals of the research group to develop a language for game design. 



